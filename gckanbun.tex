% #!lualatex
% #!ptex2pdf -l -u
%#!ptex2pdf -l
\documentclass{jlreq}\listfiles
\usepackage{gckanbun}
\usepackage{shortvrb}\MakeShortVerb{\|}

\newcommand*\pLaTeX{p\kern-.05em\LaTeX}
\newcommand*\upLaTeX{up\kern-.05em\LaTeX}
\newcommand*\LuaLaTeX{Lua\LaTeX}

\title{\textsf{gckanbun} Package}
\author{Munehiro Yamamoto}
\date{2021/03/25 v1.1}
\begin{document}
\maketitle

\begin{abstract}
This package provides a KANBUN (漢文) typesetting for (u){\pLaTeX} and \LuaLaTeX.

このパッケージは、(u){\pLaTeX}と{\LuaLaTeX}に対して漢文組みを提供します。
\end{abstract}

\section{\textsf{gckanbun}パッケージの読み込み}

\textsf{gckanbun}パッケージを読み込むだけです。

{\LuaLaTeX}の場合、\textsf{luatexja}パッケージをあらかじめ読み込んでおく必要があります。
なお、\textsf{jlreq}クラスを利用すると、jlreqクラス内部で\textsf{luatexja}パッケージを読み込みます。

\begin{quote}
\begin{verbatim}
%\usepackage{luatexja}%%<= LuaLaTeXの場合
\usepackage[<オプション>]{gckanbun}
\end{verbatim}
\end{quote}

利用可能なオプションは以下のとおりです。
\begin{itemize}
\item |prefix=<prefix>|(デフォルト値:|gckanbun|):
\textsf{gckanbun}パッケージが提供する3つのコマンド|\gckanbunruby|、|\gckanbunokurigana|、|\gckanbunkaeriten|をそれぞれ|\<prefix>ruby|、|\<prefix>okurigana|、|\<prefix>kaeriten|として提供します。
\end{itemize}

\section{例}

以下の{\LaTeX}文章を{\LuaLaTeX}でタイプセットします。
つまり、|lualatex|コマンドを実行します。

\begin{quote}
\small
\begin{verbatim}
%#!lualatex
\documentclass[tate]{jlreq}
\usepackage[prefix=]{gckanbun}%%=> <prefix>を「なし」としました。
\begin{document}
自\okurigana{リ}\kaeriten{二}桓公\kaeriten{一}八世\okurigana{ニシテ}
至\okurigana{ル}\kaeriten{二}景公\okurigana{ニ}\kaeriten{一}。
有\okurigana{リ}\kaeriten{二}晏子\okurigana{トイフ}者\kaeriten{一}、
\ruby{事}{つか}\okurigana{フ}\kaeriten{レ}之\okurigana{ニ}。
名\okurigana{ハ}嬰、\ruby{字}{あざな}\okurigana{ハ}平仲。
以\okurigana{テ}\kaeriten{二}節倹力行\okurigana{ヲ}\kaeriten{一}
重\okurigana{ンゼラル}\kaeriten{二}於齊\okurigana{ニ}\kaeriten{一}。
\ruby{一}{いっ}\ruby{狐}{こ}\ruby{裘}{きゅう}三十年、
豚肩不\kaeriten{レ}\ruby{掩}{おお}\okurigana{ハ}\kaeriten{レ}豆。
齊國之士、待\okurigana{チテ}以\okurigana{テ}擧\okurigana{グル}\kaeriten{レ}
火\okurigana{ヲ}者、七十餘家\okurigana{アリ}。
\end{document}
\end{verbatim}
\end{quote}

このとき、結果は以下のとおりです。

\begin{center}
\let\ruby\gckanbunruby
\let\okurigana\gckanbunokurigana
\let\kaeriten\gckanbunkaeriten
\hbox{\tate
\vbox to \hsize{\vfil
\begin{minipage}{15\zw}
自\okurigana{リ}\kaeriten{二}桓公\kaeriten{一}八世\okurigana{ニシテ}
至\okurigana{ル}\kaeriten{二}景公\okurigana{ニ}\kaeriten{一}。
有\okurigana{リ}\kaeriten{二}晏子\okurigana{トイフ}者\kaeriten{一}、
\ruby{事}{つか}\okurigana{フ}\kaeriten{レ}之\okurigana{ニ}。
名\okurigana{ハ}嬰、\ruby{字}{あざな}\okurigana{ハ}平仲。
以\okurigana{テ}\kaeriten{二}節倹力行\okurigana{ヲ}\kaeriten{一}
重\okurigana{ンゼラル}\kaeriten{二}於齊\okurigana{ニ}\kaeriten{一}。
\ruby{一}{いっ}\ruby{狐}{こ}\ruby{裘}{きゅう}三十年、
豚肩不\kaeriten{レ}\ruby{掩}{おお}\okurigana{ハ}\kaeriten{レ}豆。
齊國之士、待\okurigana{チテ}以\okurigana{テ}擧\okurigana{グル}\kaeriten{レ}
火\okurigana{ヲ}者、七十餘家\okurigana{アリ}。
\end{minipage}
\vfil}}%
\end{center}

\end{document}
