%% https://gist.github.com/munepi/5c9e53936e3540b8374b4ab14e6b7832
%#!lualatex
% #!ptex2pdf -l -u
\documentclass[paper=a5,tate]{jlreq}\listfiles
%\usepackage{gckanbun}%%=> provides \gckanbunruby, \gckanbunokurigana, \gckanbunkaeriten as default
\usepackage[prefix=]{gckanbun}%%=> provides \ruby, \okurigana, \kaeriten
%\usepackage[prefix=kanbun]{gckanbun}%%=> provides \kanbunruby, \kanbunokurigana, \kanbunkaeriten

\newcommand*{\onelineskip}{\par\vspace{\baselineskip}\par}
\begin{document}
【訓点(返り点・訓点送り仮名)】

% Tateten(縦点)
% Unicode	表示	名称	備考
U+3190	㆐	%IDEOGRAPHIC ANNOTATION LINKING MARK

% Kaeriten(返り点)
% Unicode	表示	名称	備考
U+3191	㆑	%IDEOGRAPHIC ANNOTATION REVERSE MARK
U+3192	㆒	%IDEOGRAPHIC ANNOTATION ONE MARK
U+3193	㆓	%IDEOGRAPHIC ANNOTATION TWO MARK
U+3194	㆔	%IDEOGRAPHIC ANNOTATION THREE MARK
U+3195	㆕	%IDEOGRAPHIC ANNOTATION FOUR MARK
U+3196	㆖	%IDEOGRAPHIC ANNOTATION TOP MARK
U+3197	㆗	%IDEOGRAPHIC ANNOTATION MIDDLE MARK
U+3198	㆘	%IDEOGRAPHIC ANNOTATION BOTTOM MARK
U+3199	㆙	%IDEOGRAPHIC ANNOTATION FIRST MARK
U+319A	㆚	%IDEOGRAPHIC ANNOTATION SECOND MARK
U+319B	㆛	%IDEOGRAPHIC ANNOTATION THIRD MARK
U+319C	㆜	%IDEOGRAPHIC ANNOTATION FOURTH MARK
U+319D	㆝	%IDEOGRAPHIC ANNOTATION HEAVEN MARK
U+319E	㆞	%IDEOGRAPHIC ANNOTATION EARTH MARK
U+319F	㆟	%IDEOGRAPHIC ANNOTATION MAN MARK

%% 以下のサンプルは、青空文庫の訓点 https://www.aozora.gr.jp/annotation/kunten.html を元にしています。
[返り点]
% 自[#二]女王國[#一]東度[#レ]海千餘里。

①
自\kaeriten{㆓}女王國\kaeriten{㆒}東度\kaeriten{㆑}海千餘里。

②レ点とその他の返り点が組み合わさっている場合
% 見[#二]大人所[#一レ]敬。

見\kaeriten{㆓}大人所\kaeriten{㆒㆑}敬。

③竪点と返り点の記載順
% 而敬‐[#二]祭天神地祇[#一]。

而敬\kaeriten-{㆓}祭天神地祇\kaeriten{㆒}。
而敬\kaeriten‐{㆓}祭天神地祇\kaeriten{㆒}。

[訓点送り仮名]
% 山城[#(ノ)]國久世[#(ノ)]郡水主[#(ニ)]坐[#(ス)]山背[#(ノ)]大國魂命[#(ノ)]神、

①
山城\okurigana{ノ}國久世\okurigana{ノ}郡水主\okurigana{ニ}坐\okurigana{ス}山背\okurigana{ノ}大國魂命\okurigana{ノ}神、

②訓点送り仮名として、漢字(万葉仮名)が用いられている場合
% 爾時倭姫命見悦給[#(弖)]、其處[#(爾)]魚見社定賜[#(支)]とあり。

爾時倭姫命見悦給\okurigana{弖}、其處\okurigana{爾}魚見社定賜\okurigana{支}とあり。

[返り点と訓点送り仮名の混在]
% 噛[#(テ)][#二]古人貧交行[#(ノ)]之詩[#(ヲ)][#一]吐[#(テ)]而戯序[#(ス)]

噛\okurigana{テ}\kaeriten{㆓}古人貧交行\okurigana{ノ}之詩\okurigana{ヲ}\kaeriten{㆒}吐\okurigana{テ}而戯序\okurigana{ス}

[例]
\ruby{一}{いっ}\ruby{狐}{こ}\ruby{裘}{きゅう}三十年

自\okurigana{リ}\kaeriten{二}
桓公\kaeriten{一}
八世\okurigana{ニシテ}
至\okurigana{ル}\kaeriten{二}
景公\okurigana{ニ}\kaeriten{一}。
有\okurigana{リ}\kaeriten{二}
晏
子\okurigana{トイフ}
者\kaeriten{一}、
\ruby{事}{つか}\okurigana{フ}\kaeriten{レ}
之\okurigana{ニ}。
名\okurigana{ハ}
嬰、
\ruby{字}{あざな}\okurigana{ハ}平仲。
以\okurigana{テ}\kaeriten{二}
節倹力
行\okurigana{ヲ}\kaeriten{一}
重\okurigana{ンゼラル}\kaeriten{二}
於
齊\okurigana{ニ}\kaeriten{一}。
\ruby{一}{いっ}
\ruby{狐}{こ}
\ruby{裘}{きゅう}
三十年、豚肩
不\kaeriten{レ}
\ruby{掩}{おお}\okurigana{ハ}\kaeriten{レ}
豆。齊國之士、
待\okurigana{チテ}
以\okurigana{テ}
擧\okurigana{グル}\kaeriten{レ}
火\okurigana{ヲ}
者、七十餘家\okurigana{アリ}。

晏子
出\okurigana{ヅ}。
其\okurigana{ノ}御之妻、
従\okurigana{リ}\kaeriten{二}
門間\kaeriten{一}
窺\okurigana{ヘバ}、
其\okurigana{ノ}夫
擁\okurigana{シ}\kaeriten{二}
大蓋\okurigana{ヲ}\kaeriten{一}、
\ruby{策}{むちう}\okurigana{チ}\kaeriten{二}
駟馬\okurigana{ヲ}\kaeriten{一}、
意気揚揚\okurigana{トシテ}
自得\okurigana{ス}。
既\okurigana{ニシテ}而
帰\okurigana{ル}。
妻請\okurigana{ヒテ}\kaeriten{レ}
去\okurigana{ランコトヲ}
曰\okurigana{ハク}、
「晏子\okurigana{ハ}
身相\okurigana{トシテ}\kaeriten{二}
齊國\okurigana{ニ}\kaeriten{一}、
名顯\okurigana{ハル}\kaeriten{二}
諸侯\okurigana{ニ}\kaeriten{一}。
観\okurigana{ルニ}\kaeriten{二}
其\okurigana{ノ}
志\okurigana{ヲ}\kaeriten{一}、
\ruby{嘗}{つね}\okurigana{ニ}
有\okurigana{リ}\kaeriten{二}
以\okurigana{テ}
自\okurigana{ラ}
下\okurigana{ルコト}\kaeriten{一}。
子\okurigana{ハ}
為\okurigana{リテ}\kaeriten{二}
人\okurigana{ノ}
僕御\okurigana{ト}\kaeriten{一}、
自\okurigana{ラ}
以\okurigana{テ}
為\okurigana{ス}\kaeriten{レ}
足\okurigana{レリト}。
妾是\okurigana{ヲ}
以\okurigana{テ}
求\okurigana{ムル}\kaeriten{レ}
去\okurigana{ランコトヲ}
也\okurigana{ト}。」
御者\ruby{乃}{すなわ}\okurigana{チ}
自\okurigana{ラ}
抑損\okurigana{ス}。
晏子
怪\okurigana{ミテ}
而問\okurigana{フ}\kaeriten{レ}
之\okurigana{ヲ}。
以\okurigana{テ}\kaeriten{レ}
實\okurigana{ヲ}
\ruby{対}{こた}\okurigana{フ}。
薦\okurigana{メテ}
為\okurigana{セリ}\kaeriten{二}
大夫\okurigana{ト}\kaeriten{一}。

\onelineskip

漢文に対するふりがなは、モノルビ仕様です。
例えば、「\ruby{母}{ぼ}\ruby{字}{じ}\okurigana{ト}」。
漢文に対するふりがなをグループルビとすると、
「\ruby{母字}{ぼじ}\okurigana{ト}」となり、正しく組版されません。

\ruby{□}{ふ}\okurigana{オクリガナ}\kaeriten{一レ}
\ruby{□}{ふり}\okurigana{オクリガナ}\kaeriten{一レ}
□\okurigana{オクリ}\kaeriten{一レ}。
□\okurigana{ランコトヲ}\kaeriten{一}□□□□□
□\okurigana{ト}\kaeriten{二}、□□□□□
□\okurigana{ト}\kaeriten{三}。

\end{document}
